\addsec{Vorlesung 10 - Sicherungsschicht}

\minisec{In welche zwei Sublayer kann der Data-Link Layer (Schicht 2 in ISO/OSI) unterteilt werden? Was sind grob die Aufgaben dieser zwei Sublayer?}
\begin{itemize}
    \item LLC Sublayer: Das Protokoll LLC fügt einem gegebenen Datenpaket aus einer übergeordneten Schicht (meist der OSI-Schicht 3 „Vermittlungsschicht“) drei Felder hinzu:
    \begin{itemize}
        \item zwei jeweils 8 Bit bzw. 1 Byte große Kennzeichen:
        \begin{itemize}
            \item DSAP (Destination Service Access Point: Einsprungadresse des Empfängers)
            \item SSAP (Source Service Access Point: Einsprungadresse des Absenders)
        \end{itemize}
        \item ein 1 oder 2 Byte großes Feld Control mit Steuerinformationen für Hilfsfunktionen wie z. B. die Datenflusssteuerung
    \end{itemize}
    \item MAC Sublayer:
    \begin{itemize}
        \item Das \textbf{Kanalzugriffsprotokoll} beschreibt, nach welchen Regeln auf ein Übertragungsmedium zugegriffen werden darf, d. h. ein Rahmen auf der Verbindungsleitung übertragen werden darf
        \item Bei \underline{Punkt-zu-Punkt}-Verbindungen ist das Leitungszugriffsprotokoll einfach.
        \item Der Sender kann einen Rahmen senden, wann immer die Verbindungsleitung frei ist (bei vollduplex immer)
        \item Bei \underline{Multi-Access-Netzen} teilen sich mehrere Teilnehmer eine Verbindungsleitung, z.B. nach dem Bus-Prinzip.
        \item Hier übernimmt der MAC Layer die Koordination der Leitungsnutzung
        \item Der MAC-Layer liefert eine eindeutige Kennung für jedes Netzwerkgerät bzw. jede Netzwerk-Karte → \textcolor{blue}{MAC} Adresse
    \end{itemize}
\end{itemize}

\minisec{Welche Übertragungsmedien (auf Schicht 1 in ISO/OSI) werden heute typischerweise eingesetzt?}
\begin{itemize}
    \item Kupferdoppelader (Twisted Pair)
    \item Koaxialkabel
    \item Funk
    \item Glasfaser
\end{itemize}

\minisec{Welche physikalischen Größen werden dabei verwendet, und wie können diese moduliert werden?}
\begin{itemize}
    \item Spannung
    \item Elektromagnetische Wellen (Funk, Licht)
    \item Modellierung über:
    \begin{itemize}
        \item Amplitude
        \item Frequenz
        \item Phase
    \end{itemize}
    \item Die Veränderung dieser Eigenschaften im Rahmen
    der Datenübertragung nennt man \textcolor{blue}{Modulation}
    \begin{itemize}
        \item Amplitudenmodulation (AM)
        \item Frequenzmodulation (FM)
        \item Phasenmodulation (PM)
    \end{itemize}
\end{itemize}

\minisec{Welche unterschiedlichen Arten von 'twisted pair'-Kabeln kennen Sie?}
\begin{itemize}
    \item Unterscheidung nach Kategorie:
    \begin{itemize}
        \item \textcolor{blue}{Kategorie 3 :} Gemeinsame Umhüllung für vier Kupferdoppeladern
        \item \textcolor{blue}{Kategorie 5 :}
        Wie Kategorie 3, aber mehr Windungen/cm (weitere Reduktion der elektromagnetischen Interferenzen Umhüllung besteht aus Teflon (bessere Isolierung, Qualität der Signale bleibt auf längere Strecken akzeptabel)
        \item \textcolor{blue}{Kategorie 6,7 :} Die Paare sind zusätzlich einzeln mit Silberfolie umwickelt
    \end{itemize}
    \item Unterscheidung nach Abschirmung:
    \begin{itemize}
        \item \textcolor{blue}{UTP Kabel (Unshielded Twisted Pair) :} Keine Abschirmung des Kabels
        \textcolor{blue}{STP Kabel (Shielded Twisted Pair) :} Abschirmung des Kabels, dadurch günstigere Eigenschaften, trotzdem in der Praxis oft UTP
    \end{itemize}
    \item Beispiele: S/UTP-Kabel (cat 5), S/STP-Kabel (cat 7)
\end{itemize}

\minisec{Wie ist ein Glasfaserkabel prinzipiell aufgebaut? Wo findet die 'Übertragung' statt?}
\begin{itemize}
    \item Von innen nach außen:
    \begin{enumerate}
        \item Faserkern / Kernglas (core)
        \item Mantelglas (cladding)
        \item Beschichtung (coating)
        \item Kunststoffummantelung (buffering)
        \item Schutzmantel
    \end{enumerate}
    \item Im Glaskern findet die Übertragung statt
\end{itemize}

\minisec{Erklären Sie im Zusammenhang mit Glasfaserkabeln die Begriffe 'Moden' und 'Dispersion'!}
\begin{itemize}
    \item Dispersion
    \begin{itemize}
        \item Begrenzt Übertragungsstrecke
        \item Lichtpuls besteht aus mehreren Wellen (Strahlen) -> Einfallswinkel dieser Strahlen unterschiedlich
        \item Lichtstrahlen kommen im Medium unterschiedlich schnell vorwärts:
        \begin{itemize}
            \item Wege (Moden) der Strahlen unterschiedlich lang (abhg. von Einfallswinkel)
            \item Strahlen eines Impulses kommen zeitversetzt am Ende des Kabels an
            \item Intensität der Impulse nimmt ab, benachbarte Impulse verschwimmen
        \end{itemize}
        \item (Weitere Faktoren können ebenso Dispersion verursachen)
    \end{itemize}
\end{itemize}

\minisec{Wie unterscheiden sich Monomode- und Multimode Glasfaserkabel? Welche Eigenschaften resultieren aus dem unterschiedlichen Aufbau?}
\begin{itemize}
    \item Monomode-Faser
    \begin{itemize}
        \item Kerndurchmesser: 8--10 $\mu m$
        \item Alle Strahlen können nur noch einen Weg nehmen
        \item Keine Dispersion (homogene Signalverzögerung)
        \item 50 $\frac{GBit}{s}$ über 100 km
        \item Teuer wegen geringem Kerndurchmesser
    \end{itemize}
    \item Multimode-Faser mit Stufenindex
    \begin{itemize}
        \item Kerndurchmesser: 50 $\mu m$
        \item Unterschiedliche Wege für Lichtwellen, je nach Einfallswinkel
        \item Starke Dispersion
        \item Bis zu 1 km
    \end{itemize}
    \item Multimode-Faser mit Gradientenindex
    \begin{itemize}
        \item Kerndurchmesser: 50 $\mu m$
        \item Brechungsindex ändert sich fließend
        \item Leicht unterschiedliche Wege für Lichtwellen
        \item Geringe Dispersion
        \item Bis zu 30 km
    \end{itemize}
\end{itemize}

\minisec{Bei welcher Netztopologie gibt es ein 'gemeinsames' Medium, auf das alle Teilnehmer zugreifen?}
\begin{itemize}
    \item Multi-Access-Netz (gemeinsames Medium)
    \begin{itemize}
        \item Nur in lokalen Netzen verwendet
        \item Alle Stationen sind an ein einziges Medium angeschlossen
        \item Sendet eine Station Daten, werden sie an alle Stationen ausgeliefert
        \item Jeder Rechner kontrolliert jedes Paket, ob es für ihn bestimmt ist
    \end{itemize}
\end{itemize}

\minisec{Bei welchen Netztopologien gibt es ausschließlich Punkt-zu-Punkt-Verbindungen zwischen den Teilnehmern?}
\begin{itemize}
    \item Stern
    \item Ring
    \item Baum (mit Stern Verbindungen)
\end{itemize}

\minisec{Vergleichen Sie Ring-, Bus- und Stern-Topologie bzgl. ihrer Vor- und Nachteile!}
\begin{itemize}
    \item Ring: Point-to-Point
    \begin{itemize}
        \item Reihe von Punkt-zu-Punkt-Verbindungen
        \item Aktive Knoten: fungieren als Repeater
        \item Ausfall des gesamten Rings bei Unterbrechung einer Verbindung
        \item Ausfall des gesamten Rings bei Ausfall eines Knotens (Bypass als Abhilfe)
        \item Große Ausdehnung möglich (aufgrund der aktiven Knoten)
        \item Einfaches Einfügen neuer Knoten
        \item Variante: bidirektionaler Ring: Knoten sind durch zwei gegenläufige Ringe miteinander verbunden
    \end{itemize}
    \item Bus: Multi-Access-Netz
    \begin{itemize}
        \item + Einfach, preiswert, einfacher Anschluss neuer Knoten
        \item + Passive Ankopplung der Stationen, der Ausfall eines Knotens ist kein Problem für die anderen Knoten
        \item – Nur eine Station zu einem Zeitpunkt kann senden;
        alle anderen Stationen können nur empfangen
        \item – Begrenzung der Zahl anschließbarer Stationen
        \item – Passive Ankopplung der Stationen, daher begrenzte Ausdehnung des Busses (aber: Repeater zur Kopplung mehrerer Busse)
    \end{itemize}
    \item Stern
    \begin{itemize}
        \item Ausgezeichneter Knoten als zentrale Station
        \begin{itemize}
            \item Nachricht von Station A wird durch die zentrale Station an Station B weitergeleitet
            \item Punkt-zu-Punkt-Verbindungen (\textcolor{blue}{Switch}), oder Broadcast (\textcolor{blue}{Hub})
            \item Verwundbarkeit durch zentralen
        \end{itemize}
        \item Knoten (Redundanz möglich)
    \end{itemize}
\end{itemize}

\minisec{Wie (und auf welcher Schicht) arbeitet ein Repeater?}
\begin{itemize}
    \item Verknüpfung von zwei Netzen zur Vergrößerung der Ausdehnung
    \item Arbeitet auf der Bitebene
    \begin{itemize}
        \item Kann einkommende Signale als „0“ oder „1“ interpretieren
        \item Empfang und \textcolor{blue}{Auffrischung} des Signals – ein empfangenes Bit wird auf der anderen Seite neu als Stromimpuls codiert
        \item Kein Verstehen von Adressen höherer Schichten, alle Daten werden weitergeleitet (das Netz bleibt z.B. ein Multi-Access-Netz)
    \end{itemize}
\end{itemize}

\minisec{Erklären Sie (im Kontext eines Netzes mit twisted-pair Kabeln) den Unterschied zwischen einem Hub und einem Switch!}
\begin{itemize}
    \item Hub = „Repeater mit mehr als zwei Anschlüssen“
    \begin{itemize}
        \item Signalauffrischung wie beim Repeater
        \item An einen Anschluss kann ein einzelner Rechner oder ein ganzer Bus angeschlossen werden
        \item Multi-Access-Netz: der Hub gibt ein empfangenes Signal auf allen Anschlüssen wieder aus, praktisch wie ein "Bus" mit mehreren Anschlüssen:
        \item \textcolor{blue}{Gemeinsamer Übertragungskanal}, d. h. Stationen können nicht gleichzeitig senden und empfangen, nur eine Station auf einmal kann senden
        \item Geringe Sicherheit, da alle Stationen mithören können
    \end{itemize}
    \item Switch - Wie Hub, aber:
    \begin{itemize}
        \item Punkt-zu-Punkt-Kommunikation zwischen zwei Stationen
        \begin{itemize}
            \item Switch kann Layer-2-Adressen (MAC-Adressen) der angeschlossenen Stationen verstehen, lernt sie und kann Daten gezielt weiterleiten
            \item Stationen können gleichzeitig senden und empfangen
            \item Nur der adressierte Empfänger erhält die Daten, andere Stationen können nicht mithören
        \end{itemize}
        \item Vermeidung von Kollisionen (‚Mikrosegmentierung‘)
        \item Puffer für jeden Port
    \end{itemize}
\end{itemize}

\minisec{Was ist ein 'Layer-3 Switch'?}
\begin{itemize}
    \item Ein Layer-3-Switch ist eine Kombination aus Router und Switch.
    \item Er beherrscht nicht nur Switching, sondern auch Routing.
    \item Da Router und Switche sehr ähnlich funktionieren – sie empfangen, speichern und leiten Datenpakete weiter – ist es nur logisch beide Geräte miteinander zu kombinieren, um daraus ein Multifunktionsgerät zu machen.
\end{itemize}

\minisec{Warum gibt es in der Schicht 2 neben dem Header auch einen Trailer? Welche Felder enthält der Ethernet-Header?}
\begin{itemize}
    \item Der Trailer enthält fehlerprüfende Maßnahmen, wie beispielsweise eine Prüfsumme.
    \item Felder vom Header:
    \begin{itemize}
        \item \todo A
    \end{itemize}
\end{itemize}