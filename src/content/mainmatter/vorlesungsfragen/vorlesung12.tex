\addsec{Vorlesung 12 - Leitungscodes, WLAN}

\minisec{Erklären Sie den Begriff Signalbildung (bzw. Leitungskodierung)!}
Signalbildung ist die Umwandlung der binären Sendedaten (nach eventueller Kompression und/oder Kodierung mit Prüfsummen etc.) in physikalische Signale.
Die Signale müssen dabei nicht unbedingt binär sein, sondern können auch mehr als 2 Zustände annehmen.

\minisec{Welche Anforderungen kennen Sie, die ein guter Leitungscode erfüllen sollte?}
\begin{itemize}
    \item Möglichst hohe \textcolor{blue}{Widerstandsfähigkeit gegen Dämpfung}
    \item \textcolor{blue}{Effizienz:} möglichst hohe Übertragungsraten durch Codewörter
    \begin{itemize}
        \item binärer Code: +5V / -5V?
        \item ternärer Code: +5 V / 0V / -5V?
        \item quaternärer Code: 4 Zustände (Codierung von 2 Bit gleichzeitig)
    \end{itemize}
    \item \textcolor{blue}{Taktrückgewinnung} beim Empfänger (\textcolor{blue}{Synchronisation}), dazu möglichst
    häufige/regelmäßige Pegelwechsel
    \item Gleichstromfreiheit: positive und negative Signale treten ungefähr gleich oft auf → kein nennenswerter elektrischer Gleichstrom-Fluss
    \item Robustheit: Können längere Sequenzen von 0 und 1 noch als solche noch erkannt werden?
    Können fehlerhafte Bits erkannt werden?
\end{itemize}

\minisec{Welche Vor- und Nachteile hat ein binärer gegenüber einem quaternären Code?}
\begin{itemize}
    \item Vorteil: Ein quaternärer Code kann 2 Nutzdatenbits in einem Code-Symbol abbilden
    \item Nachteil: Die Unterscheidung von 4 verschiedenen Signalzuständen kann anfälliger gegenüber Störungen bei der Übertragung sein.
\end{itemize}

\minisec{Erklären Sie den Unterschied zwischen Basisband- und Breitband-Übertragung!}
\begin{itemize}
    \item \textcolor{blue}{Basisband:} Das Basisband ist der natürliche Frequenzbereich des Nutzsignals (untere Grenzfrequenz $f_{\min}$ gleich oder nahe bei 0 Hz).
    Die digitalen Informationen werden ‚direkt‘ in physikalische Größen übersetzt und so über die Leitung übertragen.
    Hierzu sind Kodierungsverfahren notwendig, die festlegen, wie bei der Übertragung eine 0 bzw.\ eine 1 repräsentiert werden.
    Es kann nur je ein Signal übertragen werden
    \item \textcolor{blue}{Breitband:} Die digitalen Nutzdaten werden nicht direkt übertragen, sondern einem oder mehreren hochfrequenten Trägern aufmoduliert.
    Durch die Verwendung verschiedener Trägerwellen (Frequenzen) können dann mehrere Informationen gleichzeitig übermittelt werden
\end{itemize}

\minisec{Erklären Sie den Unterschied zwischen Bit- und Baudrate!}
\begin{itemize}
    \item Wenn die Zeitdauer (Schrittdauer) eines \textcolor{blue}{Symbols} bzw.\ \textcolor{blue}{Codeelements} $T$ ist,
    ist die Schrittgeschwindigkeit \[v_s = \frac{1}{T}\] (Einheit: \textcolor{blue}{Baud}), Symbolrate
    \item Die Übertragungsgeschwindigkeit (äquivalente Bitrate) ist dann \[v_u = v_s ld n\] ($n=$ Anzahl diskreter Zustände des Codeelements)
\end{itemize}
Bei binären Codeelementen stimmen somit \textcolor{blue}{Bitrate} und \textcolor{blue}{Baudrate} (Schrittgeschwindigkeit) überein, falls nur Codeelemente für Daten übermittelt werden (es gibt auch Codeelemente für z.\ B. die Rahmenstruktur)

\minisec{Was ist der Unterschied zwischen einem NRZ- und einem RZ-Code?}
\begin{itemize}
    \item \textbf{NRZ / NRZ-L: Non-Return\_to\_Zero:}
    \begin{itemize}
        \item Kein automatisches Zurückfallen auf einen Grundpegel.
        Hier z.\ B.:
        \item 0 = negative Spannung (konstant 0V), Pegel 1
        \item 1 = positive Spannung (konstant +5V), Pegel 2
        \item \underline{Nachteil:} bei langen 0 oder 1 Folgen \underline{Taktverlust} und \underline{keine Gleichstromfreiheit}
        \item \underline{Beispiel:} UART, RS232 (serielle Schnittstellen)
    \end{itemize}
    \includegraphics[width=0.8\textwidth]{img/NRZ-L-Code}
    \item \textbf{RZ: Return to Zero (hier unipolar)}
    \begin{itemize}
        \item 0 = 0V
        \item 1 = $\frac{T}{2}$ lang 1, $\frac{T}{2}$ lang 0
        \item \underline{Vorteil:} Taktrückgewinnung bei 1-Folgen
        \item \underline{Nachteil:} Keine Gleichstromfreiheit, kein Takt bei langen 0-Folgen
        \item \underline{Beispiel:} IrDA – Fernbedienung
    \end{itemize}
    \includegraphics[width=0.8\textwidth]{img/RZ-Code}
\end{itemize}

\minisec{Welche Eigenschaften besitzt der Manchester-Code? Was bedeutet 1B2B?
Wie unterscheiden sich die Standards nach G.E. Thomas und IEEE 802.3?}
\begin{itemize}
    \item Eigenschaften:
    \begin{itemize}
        \item Lange Folgen gleicher Signale werden durch einen Pegelwechsel in
        der Mitte jedes Bits verhindert.
        Nach G.\ E. Thomas:
        \item 0 = Polaritätswechsel von negativ (-5V) nach positiv (+5V)
        \item 1 = Polaritätswechsel von positiv (+5V) nach negativ (-5V)
        \item \underline{Vorteil:} Gleichstromfrei, Taktrückgewinnung möglich
        \item \underline{Nachteil:} Doppelte Bandbreite im Vergleich zu NRZ, Bitrate $= \frac{Baudrate}{2}$
        \item \underline{Beispiel:} 10Base2
    \end{itemize}
    \includegraphics[width=\textwidth]{img/Manchester-Code}
    \item \textcolor{blue}{1B/2B-:} ein Bit wird auf zwei Symbole kodiert
\end{itemize}

\minisec{Welche Vorteile hat ein 4B/5B Code gegenüber einem 1B/2B Code?}
\begin{itemize}
    \item \underline{Nachteil des Manchester-Codes:}
    \begin{itemize}
        \item 50\% Effizienz, d.\ h. \textcolor{blue}{1B/2B-Code} (ein Bit wird auf zwei Symbole kodiert)
        Eine Verbesserung stellt der \textcolor{blue}{4B/5B-Code} dar:
        \item vier Bit werden in fünf Symbole kodiert: 80\% Effizienz
    \end{itemize}
    \item \underline{Arbeitsweise:}
    \begin{itemize}
        \item Pegelwechsel bei 1, kein Pegelwechsel bei 0 (Differentieller NRZ-Code)
        \item Kodierung von hexadezimalen Zeichen: 0, 1,\ldots, 9, A, B,\ldots, F (4 Bit)
        in 5 Bit, sodass lange Nullenblöcke vermieden werden.
        \item Auswahl der günstigsten 16 der möglichen 32 Codewörter
        (maximal 3 Nullen in Folge)
        \item Weitere 5 Bit-Kombinationen für Steuerinformationen
        \item Erweiterbar auf 1000B/1001B-Codes?
    \end{itemize}
\end{itemize}

\minisec{Welche Eigenschaften eines Trägersignals können zur Modulation verwendet werden?}
\begin{itemize}
    \item Amplitude
    \item Frequenz
    \item Signale
    \item $s(t) = A \cdot \sin(2 \cdot \pi \cdot f \cdot t + \phi)$
\end{itemize}

\minisec{Welche Möglichkeiten kennen Sie, um bei einer Breitbandübertragung die Datenrate zu erhöhen?}
\begin{itemize}
    \item Erhöhung der Bandbreite (des Frequenzbandes, auf dem übertragen wird).
    Das ist bei höheren Trägerfrequenzen i.\ d.\ R. einfacher
    \item Steigerung der in einem Abtastvorgang modulierten binären Informationen (Verwendung eines Codes mit z.\ B. 4/8/16 Bits pro Abtastung)
\end{itemize}

\minisec{Warum ist PSK weniger störanfällig als z.\ B. ASK?}
\begin{itemize}
    \item Die Amplitude unterliegt Schwächung/Dämpfung, und ist daher störanfällig
    \item Die Phase einer Schwingung ist auch bei stärkeren Störungen unverändert
\end{itemize}

\minisec{Wie unterscheiden sich QPSK und QAM?}
\begin{itemize}
    \item \textcolor{blue}{BPSK} (Binary Phase Shift Keying):
    \begin{itemize}
        \item $=$ einfaches PSK
        \item $>$ Bitwert 0: Sinuswelle
        \item $>$ Bitwert 1: invertierte Sinuswelle
        \item Niedrige Datenraten
        \item Robuste Übertragung
        \item Auch oft als differentieller Code (DBPSK)
    \end{itemize}
    \item \textcolor{blue}{QPSK} (Quaternary Phase Shift Keying):
    \begin{itemize}
        \item Zwei Bit werden gemeinsam codiert
        \item Vier unterschiedliche Phasenlagen
        \item Doppelte Datenraten verglichen mit BPSK
    \end{itemize}
\end{itemize}

\minisec{Welche Störeinflüsse gibt es bei der Datenübertragung mit Funkwellen?}
\begin{itemize}
    \item natürliche Umgebung: Gebirge, Wasser, Vegetation, Regen, Schnee
    \item künstliche Umgebung: Gebäude etc.
\end{itemize}

\minisec{Worin unterscheidet sich ein kabelgebundenes gemeinsames Medium (z.B. ein Bus bei 10Base2) von einem funkbasierten 'gemeinsamen' Medium (Luftraum bei WLAN)?}
Bei einem Kabelgebundenen Medium werden Daten an alle angeschlossenen Teilnehmer weitergeleitet.
Bei funkbasierten Techniken teilen sich auch alle Teilnehmer das gleiche Medium ('Luft'), aber je nach Abstand können entfernte Stationen sich nicht mehr 'hören' -- das Medium zwischen diesen Stationen ist praktisch unterbrochen.

\minisec{Wie groß sind typische Übertragungsraten beim WLAN?
Welche Frequenzbänder werden benutzt?}
\begin{itemize}
    \item Datenraten:
    \begin{itemize}
        \item 1, 2, 5.5, 11, 6, 9, 12, 18, 24, 36, 48, 54, \ldots MBit/s Bruttodatenrate
        \item Abhängig von Signalqualität wird die bestmögliche Datenrate gewählt
        \item Nutzdatenrate wenig mehr als die Hälfte der jeweiligen Bruttodatenrate
    \end{itemize}
    \item Frequenzbereich:
    \begin{itemize}
        \item Freies 2.4 GHz-Band (2.4 - 2.4835 GHz) ISM = Industrial – Scientific – Medical
        \item Optional 5 GHz-Band
    \end{itemize}
\end{itemize}

\minisec{Warum ist das CSMA/CD-Verfahren bei WLAN nur schwer anwendbar?}
Zentral hierbei ist das Hidden-Station-Problem.
Dies tritt auf, wenn zwei Stationen sich gegenseitig nicht wahrnehmen, aber gleichzeitig mit einer dritten Station in der Mitte kommunizieren – was unweigerlich zu Kollisionen führt.

\minisec{Erklären Sie das 'Hidden Station'-Problem bei WLAN!}
\begin{itemize}
    \item A sendet an B, C empfängt A nicht
    \item C will an B senden, stellt freies Medium fest (CS schlägt fehl)
    \item Kollision bei B, A bemerkt sie nicht (CD schlägt fehl)
    \item A ist \textcolor{blue}{hidden} (versteckt) für C
\end{itemize}

\minisec{Erklären Sie das 'Exposed-Station'-Problem bei WLAN!}
\begin{itemize}
    \item B sendet zu A, C will zu D senden
    \item C muss warten, da CS ein „besetztes“ Medium signalisiert
    \item da A aber außerhalb der Reichweite von C ist, ist dies unnötig A
\end{itemize}

\minisec{Erklären Sie grob das Vorgehen bei CSMA/CA! Wie werden Kollisionen verhindert?}
\begin{itemize}
    \item \textcolor{blue}{Carrier Sense Multiple Access with Collision Avoidance}
    \item Kollisionen können nicht erkannt werden, darum wird versucht, sie zu vermeiden
    \item Carrier Sense mit zufallsgetriebenen Backoff-Mechanismus
    \item Kollisionsvermeidung Idee:
    \begin{itemize}
        \item Vor Beginn des Sendens: \textcolor{blue}{Carrier Sense}
        \item Falls Medium frei für mindestens eine Zeit von DIFS, starte direkt mit Übertragung
        \item Falls Medium belegt: warte bei Freiwerden erneut für DIFS, wähle dann eine Backoff-Zeit vor nächsten Zugriffsversuch (\textcolor{blue}{Kollisionsvermeidung})
        \begin{itemize}
            \item Backoff-Zeit ist Vielfaches eines Zeitslots
        \end{itemize}
    \end{itemize}
    \item Kollisionsvermeidung Vorgehen:
    \begin{itemize}
        \item Falls Medium nach Ablauf der Backoff-Zeit noch immer frei,
        starte mit Übertragung
        \item Falls Medium eher belegt wird:
        \begin{itemize}
            \item Stoppe Backoff-Zähler
            \item Verwende aktuellen Wert beim nächsten Versuch weiter
        \end{itemize}
    \end{itemize}
    \item Quittierung jeder Übertragung, da Kollisionen nicht erkannt werden können
    \begin{itemize}
        \item \textcolor{blue}{Direkte} Bestätigung jedes korrekten Datenrahmens
        \begin{itemize}
            \item Wichtige Kontrollinformation, daher werden diese bereits nach SIFS ohne jegliches Backoff versendet
        \end{itemize}
    \end{itemize}
\end{itemize}