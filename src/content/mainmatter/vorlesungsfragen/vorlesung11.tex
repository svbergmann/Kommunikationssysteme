\addsec{Vorlesung 11 - Kanalzuteilung, Fehlerkorrektur}

\minisec{Welche Aufgabe hat der MAC Sublayer in der Schicht 2 (Wiederholungsfrage, siehe Vorlesung 10)?}
\begin{itemize}
    \item \todo A
\end{itemize}

\minisec{Welche Eigenschaften hätte ein ideales Mehrfachzugriffsprotokoll?}
Broadcast-Medium mit Maximalrate R bps
\begin{itemize}
    \item Ein einzelner Knoten kann mit der Rate R übertragen
    \item M Knoten können mit einer mittleren Rate von $\frac{R}{M}$ übertragen
    \item Das Protokoll ist dezentral, d.\ h. es gibt keine Master-Knoten, die ausfallen und das ganze System zum Absturz bringen können
    \item Das Protokoll ist einfach und kann somit kostengünstig implementiert werden
\end{itemize}

\minisec{Nennen Sie 3 grundsätzliche Strategien, wie der mehrfache Zugriff auf ein gemeinsames Medium gelöst werden kann!}
\begin{itemize}
    \item \textcolor{blue}{Kanalaufteilungsprotokolle (Multiplexing)}
    \begin{itemize}
        \item Aufteilung des Übertragungskanals in kleine Übertragungseinheiten (Zeitfenster, Frequenzen, Kodierung)
        \item Exklusive Nutzung der Übertragungseinheiten für einzelne Knoten
    \end{itemize}
    \item \textcolor{blue}{Zufallszugriffsprotokolle}
    \begin{itemize}
        \item Keine Unterteilung des Kanals, jeder überträgt mit der gesamten Leitungskapazität.
        Allerdings sind \underline{Kollisionen} von Nachrichten möglich und müssen korrekt behandelt werden
    \end{itemize}
    \item \textcolor{blue}{Rotationsprotokolle}
    \begin{itemize}
        \item Der Zeitpunkt, wann ein Knoten senden kann wird durch eine spezielle Koordinierung nach einem flexiblen Rotationsprinzip im Übertragungsmedium ausgetauscht.
    \end{itemize}
\end{itemize}

\minisec{Erklären Sie kurz die Funktionsweise von TDMA, FDMA und CDMA!}
\begin{itemize}
    \item \textcolor{blue}{TDMA: Time Division Multiple Access:}
    \begin{itemize}
        \item Aufteilung des Kanals in kleine Zeiteinheiten, die sich in Runden wiederholen
        \item Jeder Knoten kann eine bestimmte Anzahl an Zeiteinheiten in jeder Runde Nutzen
        \item Ungenutzte Einheiten (slots) bleiben ungenutzt
        \item Die Zeiteinheiten können meist auch gruppiert angefordert werden
        \item Anforderung entspricht Konfiguration der NIC-Karte
    \end{itemize}
    \item \textcolor{blue}{FDMA: Frequency Division Multiple Access}
    \begin{itemize}
        \item Aufteilung des Kanals in Frequenzbänder
        \item Nach Nyquist kann ein rauschfreier Kanal mit einem Frequenzband der Breite x Hz binäre Signale nicht mit mehr als 2x Bps übertragen (Nyquist-Bandbreite)
        \item Jede Station nutzt ein Frequenzband
        \item Die Kapazität nicht genutzter Frequenzbänder geht verloren
    \end{itemize}
    \item \textcolor{blue}{CDMA: Code Division Multiple Access}
    \begin{itemize}
        \item Keine Unterteilung in Zeitslots oder Frequenzen
        \item Nutzung von ‚orthogonalen Codes‘ bei der (De-) Modulation, die sich im Medium überlagern/mischen, aber trotzdem auf Empfängerseite eindeutig wiederhergestellt werden können
        \item Vorteil: Es gibt keine ungenutzten Resourcen wie bei TDMA/FDMA
    \end{itemize}
\end{itemize}

\minisec{Erklären Sie die Funktionsweise von CSMA/CD! Was bedeutet diese Abkürzung?}
\begin{itemize}
    \item \textcolor{blue}{Carrier Sense Multiple Access (CSMA)}
    \begin{itemize}
        \item höre vor der Übertragung das Medium ab
        \item sende nur, falls das Medium frei ist
        \item Analogie: Unterbreche keine Anderen
    \end{itemize}
    \item \textcolor{blue}{Carrier Sense Multiple Access with Collision Detection (CSMA/CD)}
    \begin{itemize}
        \item wie CSMA, zusätzlich: höre während der Sendung das Medium weiter ab und breche die Übertragung ab, wenn eine Kollision auftritt
        \item sende ein Jamming-Signal, damit jede Station weiß, dass eine Kollision aufgetreten ist und die Nachricht nutzlos geworden ist.
        \item Wichtig: Man muss so lange senden, dass man ein Jamming -Signal während dessen auch mitbekommt
    \end{itemize}
\end{itemize}

\minisec{Erklären Sie, warum bei CSMA/CD ein Sender eine Mindestzeit senden muss (und parallel per "CD" das Medium abhören muss)!
Wie lässt sich diese Mindestzeit berechnen und ggf. in eine Mindest-Rahmengröße umrechnen?}
\begin{itemize}
    \item
\end{itemize}

\minisec{Wie funktioniert und wozu dient der "Binary Exponential Backoff"-Algorithmus?}
Um nach einer Kollision die gleichzeitige Wiederholung der kollidierten Sendungen
zu vermeiden (Folgekollision), wird eine zufällige Wartezeit aus einem vorgegebenen
Intervall gezogen.
Das Intervall wird \textit{klein} gehalten, um große Wartezeiten
bis zur Wiederholung zu vermeiden.
Dadurch ist allerdings das Risiko eines
Folgekonflikts groß.
Kommt es so zu einer weiteren Kollision, wird das Intervall
vor dem nächsten Versuch vergrößert, um mehr Spielraum für alle sendenden
Parteien zu schaffen.

\minisec{Welche heute typischerweise eingesetzten Ethernet-Standards kennen Sie? Wie sind die Namen aufgebaut?}
Basiert auf IEEE 802.3 „CSMA/CD“
4 Klassen von Ethernet-Varianten:
\begin{itemize}
    \item Standard Ethernet → 10 $\frac{Mb}{s}$
    \item Fast Ethernet → 100 $\frac{Mb}{s}$
    \item Gigabit Ethernet → 1000 $\frac{Mb}{s}$
    \item 10Gigabit-Ethernet → 10000 $\frac{Mb}{s}$
\end{itemize}

\minisec{Warum wurde bei Gigabit-Ethernet die minimale Rahmenlänge von 64 auf 520 Bytes erhöht?}
\begin{itemize}
    \item  \todo A
\end{itemize}

\minisec{Warum wurde bei Fast-Ethernet die Segmentlänge von 2500m (10MBit Ethernet) auf 200--250m verkleinert?}
Die minimale Rahmenlänge zur Kollisionserkennung bei Ethernet beträgt 64 Byte.
Bei 100 Mb/s wird der Rahmen aber ca.\ 10 Mal so schnell abgesendet, sodass eine Kollisionserkennung nicht mehr gewährleistet ist.

\minisec{Wie ist die Rahmen-Struktur eines Ethernet-Paketes aufgebaut? Welche Daten (-Felder) werden im Header und Trailer auf Schicht 2 übertragen?}
Ethernet-Rahmen:
\begin{itemize}
    \item \textcolor{blue}{\textit{Präambel :}} kennzeichnet eine folgende Übertragung und synchronisiert den Empfänger mit dem Sender.
    \item Der \textcolor{blue}{\textit{Start-of-Frame-Delimiter :}} (bzw. die beiden aufeinanderfolgenden Einsen) zeigen an, dass endlich Daten folgen.
    \item \textcolor{blue}{\textit{Destination Address :}} das erste Bit kennzeichnet den Empfänger: entweder eine einzelne Station (1. Bit = 0) oder eine Gruppenadresse (1.\ Bit = 1; \ Broadcast ist auch hier durch 11 \ldots 1 gegeben).
    \item \textcolor{blue}{\textit{Length/Type :}} Bei einem Wert bis 1500 wird die Angabe als Länge des Datenteils aufgefasst (dies ist der Fall beim sogenannten CSMA/CD), bei einem Wert ab 1536 wird hier angegeben, an welches Schicht-3-Protokoll die Daten weitergegeben werden sollen (verwendet bei Ethernet).
    \item \textcolor{blue}{\textit{FCS :}} Checksumme, 32-Bit CRC. Diese erstreckt sich über die Felder DA, SA, Length/Type, Data/Padding.
\end{itemize}
Felder in Schicht 2:
\begin{itemize}
    \item MAC-Empfänger
    \item MAC-Absender
    \item 802.1Q-Tag (opt.)
    \item EtherType
    \item Nutzlast (1500 bytes)
    \item Frame Check Sequence
\end{itemize}

\minisec{Erklären Sie die grundsätzliche Vorgehensweise des Medienzugriffes beim Token-Ring Netz!}
\begin{itemize}
    \item “Token“-Verfahren, nur wer ein bestimmtes Token ( = Bitfolge) besitzt, darf senden
    \item Die Rechner teilen sich einen Ring aus Punkt-zu-Punkt-Verbindungen
    \item Das Token wird zyklisch weitergegeben
\end{itemize}

\minisec{Was ist ein Hamming-Abstand bei einem Code C mit den Wörtern $c_1$ bis $c_n$?}
\begin{itemize}
    \item Anzahl unterschiedlicher Bits zweier Codewörter $c_1$ und $c_2$, d.\ h. Anzahl der 1-Bits von $c_1$ XOR $c_2$.
    \item Beispiel: d(10001001, 10110001) = 3
    \item Hamming-Abstand D von vollständigem Code C : \[ D(c)\coloneqq \min\{d(c_1 ,c_2)\ |\ c_1 ,c_2\ \epsilon \ C; \ c_1 \neq c_2\}\]
\end{itemize}

\minisec{Was versteht man unter "Forward Error Correction"?}
\begin{itemize}
    \item \todo A
\end{itemize}

\minisec{Wie groß muss der Hamming-Abstand mindestens sein, um e-Bitfehler zu erkennen bzw. zu beheben?}
\begin{itemize}
    \item \textcolor{blue}{erkennen} von e-Bit-Fehlern: Hamming-Abstand \textcolor{blue}{e+1} notwendig
    \item \textcolor{blue}{beheben} von e-Bit-Fehlern: Hamming-Abstand \textcolor{blue}{2e+1} notwendig
\end{itemize}

\minisec{Nennen Sie einen Code, der mit einer minimalen Anzahl von Prüfbits Einzelbitfehler korrigieren kann!}
\begin{itemize}
    \item \todo A
\end{itemize}

\minisec{Was versteht man unter Modulo-2 Arithmetik? Wie werden Addition und Subtraktion durchgeführt?}
\begin{itemize}
    \item Die Rechenoperationen werden in der Modulo-2-Arithmetik einfacher, da hierbei keine Überträge zu berücksichtigen sind!
    \item Addition und Subtraktion führen so zu dem gleichen Ergebnis.
    \item Wir können einfach mit XOR arbeiten!
    \item Digitaltechnik, hier Halbaddierer: Eine Addition ist logisch ein XOR, das Carry ein UND
    \item Es gibt eine Bitkombination n, sodass gilt: $(D \cdot 2^r) XOR \ R = n G$
    \item D.h., R soll so gewählt werden, dass G in $D \cdot 2^r$ ohne Rest teilbar ist
    \item $D \cdot 2^r = nG \ XOR \ R$
    \item Damit kann man R berechnen, denn wenn man $D \cdot 2^r$ durch G teilt, ist der Rest des Wertes genau R
    \[R = Rest\left[\frac{D \cdot 2^r}{G}\right]\]
\end{itemize}

\minisec{Was leistet bzw. wie funktioniert das CRC-Verfahren? Wozu dient das Generator-Polynom?
Wo wird das Verfahren im Kontext von Ethernet eingesetzt?}
\begin{itemize}
    \item Polynom-Codes: Interpretiere Datenbits D als Bitkette eines Polynoms, dessen Koeffizienten d0--10-1-Werte der Bitkette sind
    \item Prüfung basiert auf Polynom-Arithmetik
    \item Sender und Empfänger einigen sich auf ein gemeinsam verwendetes Bitmuster der Länge r+1 Bit, das als Generator G bezeichnet wird. Das höchstwertige Bit hiervon ist 1
    \item Konzept: Berechne die r CRC-bits R so, dass die d + r Bits (als Binärzahl interpretiert) mit der Modulo-2-Arithmetik genau durch G teilbar sind
    \begin{itemize}
        \item Empfänger kennt G, teilt <D,R> durch G. Falls der Rest ungleich 0, so liegt ein Fehler vor!
        \item Kann Burst-Fehler von weniger als r+1 Bits und jede ungerade Fehlerzahl erkennen
    \end{itemize}
\end{itemize}
