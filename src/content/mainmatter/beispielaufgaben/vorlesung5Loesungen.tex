\addsec{Vorlesung 5 - Lösungen}

\minisec{Übungsaufgabe 1:}
Die IDs können frei gewählt werden, müssen sich nur unterscheiden:

\bigskip

\begin{minipage}[c]{0.45\textwidth}
    A → MTU 2000 → R
    \centering
    \begin{tabular}{|c|c|c|c|}
        \hline
        ID & MF & Total Length & Offset \tabularnewline
        \hline
        \textbf{1} & \textbf{1} & \textbf{1996} & \textbf{0} \tabularnewline
        \hline
        \textbf{2} & \textbf{0} & \textbf{1644} & \textbf{247} \tabularnewline
        \hline
    \end{tabular}
\end{minipage}
\begin{minipage}{0.5\textwidth}
    R → MTU 1500 → B
    \centering
    \begin{tabular}{|c|c|c|c|}
        \hline
        ID & MF & Total Length & Offset \tabularnewline
        \hline
        \textbf{11} & \textbf{1} & \textbf{1500} & \textbf{0} \tabularnewline
        \hline
        \textbf{12} & \textbf{1} & \textbf{516}  & \textbf{185} \tabularnewline
        \hline
        \textbf{21} & \textbf{1} & \textbf{1500} & \textbf{247} \tabularnewline
        \hline
        \textbf{22} & \textbf{0} & \textbf{164}  & \textbf{432} \tabularnewline
        \hline
    \end{tabular}
\end{minipage}

\medskip

Auf der ersten Strecke (A → R) wird das Paket in zwei Unterpakete mit einer Payload-Größe von 1976 + 1624 = 3600 Bytes zerlegt.
1976 muss dabei durch 8 teilbar sein!
Auf der zweiten Strecke (R → B) werden diese zwei Pakete dann in Pakete mit Payload-Größe 1480 + 496 + 1480 + 144 = 3600 Bytes zerlegt.
1480 muss dabei durch 8 teilbar sein!
Das letzte Paket muss das MF-Bit (\textbf{M}ore \textbf{F}ragments) auf 0 setzen!